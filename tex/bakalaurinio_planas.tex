\documentclass{VUMIFPSbakalaurinis}
\usepackage{algorithmicx}
\usepackage{algorithm}
\usepackage{algpseudocode}
\usepackage{amsfonts}
\usepackage{amsmath}
\usepackage{bm}
\usepackage{caption}
\usepackage{color}
\usepackage{float}
\usepackage{graphicx}
\usepackage{listings}
\usepackage{subfig}
\usepackage{wrapfig}
\usepackage{enumitem}
\addbibresource{./bibliografija_planas.bib}


% Titulinio aprašas
\university{Vilniaus universitetas}
\faculty{Matematikos ir informatikos fakultetas}
\institute{Informatikos institutas}  % Užkomentavus šią eilutę - institutas neįtraukiamas į titulinį
\department{Programų sistemų bakalauro studijų programa}
\papertype{Bakalauro baigiamojo darbo planas}
\title{Krepšinio taisyklių pažeidimo atpažinimas}
\titleineng{Basketball rules violation recognition}
\author{Lukas Cedronas}
% \reviewer{Vardenis Pavardenis} % recenzentas
% \secondauthor{Vardonis Pavardonis}   % Pridėti antrą autorių
\supervisor{partn. prof., dr. Vytautas Ašeris}
\date{Vilnius – \the\year}

% Nustatymai
% \setmainfont{Palemonas}   % Pakeisti teksto šriftą į Palemonas (turi būti įdiegtas sistemoje)
% \bibliography{aaaa.bib}

\begin{document}
\maketitle

%% Padėkų skyrius
% \sectionnonumnocontent{}
% \vspace{7cm}
% \begin{center}
%     Padėkos asmenims ir/ar organizacijoms
% \end{center}

\section{Darbo planas}

\subsection{Tyrimo objektas ir aktualumas}
Krepšinis - dinamiškas ir intensyvus sportas, kuriame galimi veiksmai su kamuoliu yra apriboti taisyklių, tokių kaip dvigubo varymo, žingsnių taisyklės ir daug kitų. Tai lemia, jog teisėjams neretai tampa sunku teisingai įvertinti, kada buvo pažeistos taisyklės, todėl atsiranda poreikis pasinaudoti kompiuterio pagalba. Žmogaus judesių atpažinimo ir sekimo problematikai galima pasinaudoti kompiuterinės regos ir dirbtinio intelekto technologijomis. Kompiuterinės regos (\textit{angl.} computer vision) metodai yra aktualūs kompiuterizuojant sudėtingų vaizdinių uždavinių sprendimus, taip padedant išvengti žmogiškųjų klaidų. Darbe bus analizuojami ir įgyvendinami kompiuterinės regos metodai ir algoritmai, padedantys atpažinti, kada krepšinio žaidėjas pažeidė žingsnių bei dvigubo varymo taisykles. 

\subsection{Darbo tikslas}
Darbo tikslas - pasiūlyti žingsnių bei dvigubo varymo taisyklių pažeidimo atpažinimo algoritmą ir jį realizuojančią programinę įrangą naudojantis kompiuterinės regos metodais.

\subsection{Uždaviniai ir laukiami rezultatai}
Darbo uždaviniai:
\begin{itemize}[topsep=5pt,itemsep=-1ex,partopsep=2ex,parsep=2ex]
 \item Išanalizuoti ir palyginti galimus metodus žmogaus kūno dalims vaizdinėje medžiagoje atpažinti.
 \item Apibrėžti ir įgyvendinti algoritmus, nustatytu tikslumu atpažįstančius žingsnių taisyklės pažeidimą trimatėje erdvėje iš skirtingų kampų.
 \item Apibrėžti ir įgyvendinti algoritmus, nustatytu tikslumu sugebančius atskirti dvigubo varymo taisyklės pažeidimą nuo kitų veiksmų, nepažeidžiančių taisyklės.
 \item Palyginti įgyvendintų algoritmų su vaizdine medžiaga korektiškumą. 
\end{itemize}

Laukiami rezultatai:
\begin{itemize}[topsep=5pt,itemsep=-1ex,partopsep=2ex,parsep=2ex]
 \item Palyginti galimi kompiuterinės regos metodai.
 \item Apibrėžtas algoritmas, atpažįstantis žmogaus kūno dalis.
 \item Apibrėžtas algoritmai, atpažįstantys žingsnių bei dvigubo varymo taisyklių pažeidimus.
 \item Parašyta programinė įranga, įgyvendinanti algoritmus.
\end{itemize}

\subsection{Tyrimo metodas}
Darbe kompiuterinės regos metodai bus pritaikomi žaidėjo kūno dalių (rankų, kojų) bei kamuolio atpažinimui vaizdinėje medžiagoje. Sukūrus žaidėjo matematinį modelį bus pritaikomi taisyklių pažeidimo algoritmai, dvimatėje matricoje pagal žaidėjo, jo kūno dalių ir kamuolio poziciją suskaičiuojantys atliktus žingsnius ir ryšį su kamuoliu. Taip pat bus sukaupta vaizdinė medžiaga su įvairių taisyklių pažeidimais ir be jų. Sukurtos programinės įrangos pagalba bus atpažintos pažeistos krepšinio taisykles bei išvesti rezultatai.

\subsection{Darbo atlikimo procesas}
Darbas bus pradedamas literatūros šaltinių analize. Vėliau bus išanalizuoti ir pasirinkti tinkami kompiuterinės regos metodai, kuriais remiantis bus apibrėžti taisyklių pažeidimo algoritmai. Algoritmai bus įgyvendinami programine įranga, parašyta Python kalba, naudojantis OpenCV biblioteka. Giliojo mokymo bibliotekos, tokios kaip Tensorflow ar Keras, taip pat bus apsvarstytos ir pasirinktos išnagrinėjus galimus žmonių atpažinimo erdvėje algoritmus. Taip pat bus nufilmuota vaizdinė medžiaga, kurioje žaidėjai atliks tam tikrus krepšinio veiksmus, dalis iš jų pažeis taisykles, dalis - ne. Vaizdo medžiaga bus paduota į programą kaip įvesties informacija, gauti rezultatai išanalizuoti ir pasiektos išvados. 

\subsection{Darbui aktualūs literatūros šaltiniai}

Darbe bus nagrinėjami kompiuterinės regos algoritmai objektų atpažinimui. Dominančio objekto išskyrimui nuo fono galima pasinaudoti segmentavimu pagal spalvą \cite{VERHES_LAHI_2005} \cite{SzeliskiCompVision}. Taip pat buvo pasiūlytas būdas atlikti judančio kūno segmentavimą naudojantis ne nuo spalvų priklausančiomis technikomis \cite{4717823}. Segmentavimo metu išskirtiems objektams galima pritaikyti morfologines transformacijas triukšmo pašalinimui \cite{4767941}. Vaizdų atpažinimo problemas galima spręsti ir pasinaudojus neuroniniais tinklais \cite{KhanConvVision}. Krepšininko kūno atpažinimas naudojantis neuroniniais tinklais gali būti problematiškas dėl intensyvaus judėjimo \cite{THOMAS20173}, tačiau tinkamai pasirinkus pradinių duomenų aibę žmogaus kūno ir jo pozą galima atpažinti tiksliai ir efektyviai \cite{Chen_2018_CVPR}. Kamuoliui atpažinti galima pasinaudoti dydžio, formos ir kompaktiškumo filtrais \cite{6224370}. Darbe algoritmai bus įgyvendinami OpenCV biblioteka, puikiai tinkančia kompiuterinės regos algoritmams įgyvendinti \cite{BradskiOpenCV}, tuo tarpu objektų atpažinimui naudojantis giliojo mokymo metodais galima pasinaudoti TensorFlow biblioteką \cite{ShanDeepVision} \cite{HuaMachineLearning}.

\printbibliography[heading=bibintoc]  % Šaltinių sąraše nurodoma panaudota
% literatūra, kitokie šaltiniai. Abėcėlės tvarka išdėstomi darbe panaudotų
% (cituotų, perfrazuotų ar bent paminėtų) mokslo leidinių, kitokių publikacijų
% bibliografiniai aprašai. Šaltinių sąrašas spausdinamas iš naujo puslapio.
% Aprašai pateikiami netransliteruoti. Šaltinių sąraše negali būti tokių
% šaltinių, kurie nebuvo paminėti tekste. Šaltinių sąraše rekomenduojame
% necituoti savo kursinio darbo, nes tai nėra oficialus literatūros šaltinis.
% Jei tokių nuorodų reikia, pateikti jas tekste.

\end{document}
