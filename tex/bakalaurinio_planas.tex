\documentclass{VUMIFPSbakalaurinis}
\usepackage{algorithmicx}
\usepackage{algorithm}
\usepackage{algpseudocode}
\usepackage{amsfonts}
\usepackage{amsmath}
\usepackage{bm}
\usepackage{caption}
\usepackage{color}
\usepackage{float}
\usepackage{graphicx}
\usepackage{listings}
\usepackage{subfig}
\usepackage{wrapfig}
\usepackage{enumitem}

% Titulinio aprašas
\university{Vilniaus universitetas}
\faculty{Matematikos ir informatikos fakultetas}
\institute{Informatikos institutas}  % Užkomentavus šią eilutę - institutas neįtraukiamas į titulinį
\department{Programų sistemų bakalauro studijų programa}
\papertype{Bakalauro baigiamojo darbo planas}
\title{Krepšinio taisyklių pažeidimo atpažinimas}
\titleineng{Basketball rules violation recognition}
\author{Lukas Cedronas}
% \reviewer{Vardenis Pavardenis} % recenzentas
% \secondauthor{Vardonis Pavardonis}   % Pridėti antrą autorių
\supervisor{partn. prof., dr. Vytautas Ašeris}
\date{Vilnius – \the\year}

% Nustatymai
% \setmainfont{Palemonas}   % Pakeisti teksto šriftą į Palemonas (turi būti įdiegtas sistemoje)
\bibliography{bibliografija}

\begin{document}
\maketitle

%% Padėkų skyrius
% \sectionnonumnocontent{}
% \vspace{7cm}
% \begin{center}
%     Padėkos asmenims ir/ar organizacijoms
% \end{center}

\section{Darbo planas}

\subsection{Tyrimo objektas ir aktualumas}
Kompiuterinės regos (\textit{angl.} computer vision) technologijos tampa vis svarbesnės, mokslui ir technologijų gigantams atkreipiant dėmesį į dirbtinį intelektą ir su juo susijusius metodus, siekiant sudėtingus darbus algoritmizuoti, taip išvengiant žmogiškųjų klaidų. Krepšinis - vienas iš sportų, galintis gauti daug naudos iš kompiuterinės regos ir dirbtinio intelekto research: sporto intensyvumas ir dinamiškumas lemia tai, jog teisėjams neretai tampa sunku teisingai įvertinti, kada buvo pažeistos taisyklės. Šiame darbe bus analizuojami ir įgyvendinami kompiuterinės regos metodai ir algoritmai, padedantys atpažinti, kada krepšinio žaidėjas pažeidė žingsnių bei dvigubo varymo taisykles. 

\subsection{Darbo tikslas}
Darbo tikslas - sukurti žingsnių bei dvigubo varymo taisyklių pažeidimo atpažinimo programinę įrangą, įgyvendinančią vaizdo atpažinimo algoritmus naudojantis kompiuterinės regos metodais.

\subsection{Uždaviniai ir laukiami rezultatai}
Darbo uždaviniai:
\begin{itemize}[topsep=5pt,itemsep=-1ex,partopsep=2ex,parsep=2ex]
 \item Išanalizuoti ir palyginti galimus metodus žmogaus kūno dalims vaizdinėje medžiagoje atpažinti.
 \item Apibrėžti ir įgyvendinti algoritmus, atpažįstančius žingsnių taisyklės pažeidimą trimatinėje erdvėje iš skirtingų kampų.
 \item Apibrėžti ir įgyvendinti algoritmus, sugebančius atskirti dvigubo varymo taisyklės pažeidimą nuo kitų veiksmų, nepažeidžiančių taisyklės.
 \item Pravaliduoti ir palyginti įgyvendintus algoritmus su vaizdine medžiaga ir išanalizuoti rezultatus.
\end{itemize}

Laukiami rezultatai:
\begin{itemize}[topsep=5pt,itemsep=-1ex,partopsep=2ex,parsep=2ex]
 \item Išnagrinėta naujausia kompiuterinės regos medžiaga, išnagrinėti bei palyginti galimi kompiuterinės regos metodai.
 \item Apibrėžtas algoritmas, atpažįstantis žmogaus kūno dalis.
 \item Apibrėžtas algoritmai, atpažįstantys žingsnių bei dvigubo varymo taisyklių pažeidimus.
 \item Algoritmai įgyvendinti sukuriant programinę įrangą.
\end{itemize}

\subsection{Tyrimo metodas}
Tyrimo metodas - sukaupti vaizdinę medžiagą su įvairių taisyklių pažeidimais ir be jų ir leisti sukurtai programinei įrangai atpažinti pažeistas krepšinio taisykles, išvedant teisingų ir neteisingų atpažinimų rezultatus.

\subsection{Darbo atlikimo procesas}
Darbas bus pradedamas literatūros šaltinių analize. Vėliau bus išanalizuoti ir pasirinkti tinkami kompiuterinės regos metodai, kuriais remiantis bus apibrėžti taisyklių pažeidimo algoritmai. Algoritmai bus įgyvendinami programine įranga, parašyta Python kalba, naudojantis OpenCV ir kitomis susijusiomis bibliotekomis. Taip pat bus nufilmuota vaizdinė medžiaga, kurioje žaidėjai atliks tam tikrus krepšinio veiksmus, dalis iš jų pažeis taisykles, dalis - ne. Vaizdo medžiaga bus paduota į programą kaip įteiktis, gavus rezultatus - jie bus išanalizuoti ir pasiektos išvados. 

\subsection{Literatūros šaltiniai}
Darbui aktualūs literatūros šaltiniai: Computer Vision: Models, Learning, and Inference By Simon J. D. Prince, Computer Vision: Algorithms and Applications By Richard Szeliski, 
Learning OpenCV: Computer Vision with the OpenCV Library By Gary Bradski, Adrian Kaehler, A Guide to Convolutional Neural Networks for Computer Vision
By Salman Khan, Hossein Rahmani, Syed Afaq Ali Shah, Mohammed Bennamoun, Deep Learning for Computer Vision: Expert techniques to train advanced ...
By Rajalingappaa Shanmugamani, Machine Learning with TensorFlow 1.x by Quan Hua, Shams Ul Azeem, Saif Ahmed




\end{document}
